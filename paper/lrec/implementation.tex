\section{Implementation}
\label{sec:implementation}

In this section, we describe the implementation of the tool.
The SAT-encoding we use is similar to the one introduced in \cite{listenmaa_claessen2015}, with one key difference: in this paper, we operate on {\em symbolic sentences} instead of concrete sentences from a corpus. The idea is that the SAT-solver is going to find the concrete sentence for us.

\paragraph{Preliminaries}

Our analysis operates on one rule $R$ at a time, and is concerned with answering the following question: ``Does there exist an input sentence $S$ that can trigger rule $R$, even after passing all rules $R'$ that came before $R$?''

Before we can do any analysis any of the rules, we need to find out what the set of all possible readings of a word is. We can do this by extracting this information from a lexicon, but there are other ways too. In our experiments, this number has ranged from about 300 to about 6000. 

Furthermore, when we analyze a rule $R$, we need to decide the {\em width} of the rule $R$: How many different words should there be in a sentence that can trigger $R$? We denote this value by $w(R)$. Most often, $w(R)$ can be easily determined by looking at how far away the rule context indexes in the sentence relative to the target. For example, in the rule mentioned in the introduction, the width is 2.

If the context contains a \verb!*!, we may need to make an approximation of $w(R)$ which may result in false negatives later on in the analysis.

\paragraph{Symbolic sentences}

The SAT-solver comes in when we start looking at the rules and what they do to sentences. We start each analysis by creating a so-called {\em symbolic sentence}, which is our representation of the sentence $S$ we are looking for. A symbolic sentence is a sequence of {\em symbolic words}, and a symbolic word is a table of all possible readings that a word can have, each reading paired up with a SAT-variable.

The length of the symbolic sentence we create when we analyze a rule $R$ is $w(R)$. For the rule in the introduction, we have $w(R)=2$ and a symbolic sentence may look as follows:
\begin{center}
\begin{tabular}{c|c|c}
word1 & word2 & reading \\
\hline
$v_1$ & $w_1$ & det def \\
$v_2$ & $w_2$ & noun sg \\
$v_3$ & $w_3$ & noun pl \\
$v_4$ & $w_4$ & verb sg \\
$v_5$ & $w_5$ & verb pl \\
\end{tabular}
\end{center}
Here, $v_i$ and $w_j$ are SAT-variables belonging to word1 and word2, respectively. We can see that the possible number of readings here was 5.

The SAT-solver may contain extra constraints about the variables. For example, we would like the input sentence to have at least one reading per word, so we add the following two constraints:
\begin{center}
\begin{tabular}{c}
$v_1 \vee v_2 \vee v_3 \vee v_4 \vee v_5$, \\
$w_1 \vee w_2 \vee w_3 \vee w_4 \vee w_5$ \\
\end{tabular}
\end{center}
The point of the symbolic sentence is that any solution that is produced by the SAT-solver represents an input sentence that we are looking for.

\paragraph{Applying a rule}

In order to say something about the rules in the grammar, we need to be able to apply a given rule $R'$ to the symbolic sentence, resulting in a new sentence. The way this works is that we investigate the rule, and generate a new sentence based on what changes can be made to the sentence.

For example, if we apply the rule from the introduction to the symbolic sentence above, we generate the following resulting symbolic sentence:
\begin{center}
\begin{tabular}{c|c|c}
word1 & word2 & reading \\
\hline
$v_1$ & $w_1$ & det def \\
$v_2$ & $w_2$ & noun sg \\
$v_3$ & $w_3$ & noun pl \\
$v_4$ & $w_4'$ & verb sg \\
$v_5$ & $w_5'$ & verb pl \\
\end{tabular}
\end{center}
The rule can only affect readings of word2 that have a verb tag, so we create two new variables $w_4'$ and $w_5'$. We also have to say what their values are, which we do by adding constraints to the SAT-solver. The constraint we add for $w_4'$ is:
\begin{center}
\begin{tabular}{c}
$w_4' \Leftrightarrow [ w_4 \wedge \neg{}(v_1 \wedge (w_1 \vee w_2 \vee w_3)) ]$ \\
\end{tabular}
\end{center}
In other words $w_4'$ can only be true if $w_4$ was true already, and the rule has not triggered. The rule triggers if $v_1$ is true and one of word2's readings is not verb.

EXPLAIN THIS BETTER


\paragraph{Putting it all together}


-- JUNK BELOW --

We use a SAT-solver to construct such a sentence $S$, or to prove the non-existence of such a sentence. The way we do this is by first constructing a {\em symbolic sentence} $S_0$, a sentence of a particular length where all words in principle can have any possible reading, and where each reading for each word is represented by a SAT variable. $S_0$ represents all possible input sentences.

An example 


We then proceed by calculating a symbolic representation of all sentences that could exist that have passed applying each rule $R'$ that preceeds the rule $R$ that is under analysis to the symbolic sentence, in the same sequence as they appear in the grammar. After each such application, we get a new symbolic sentence that represents all possible sentences that could be the result after applying those rules. 







%%% FROM INARI %%%

\paragraph{What kind of questions we can ask?}

The example so far has been of type ``Can rule Z apply after A-Y?''
In case a conflict is detected, we can dig into it and refine where the problem is: e.g. is the condition not met, or is it the case that the target is already removed or the only remaining analysis.
We can find what other rule or rules prevent others from applying, or if the conflict is rule-internal, such as nonexisting tag set or contradicting requirements of a context word (e.g. a word must be unambiguously two different POS).


In addition to analysing a whole grammar, 
we can construct all kinds of symbolic sentences to test out individual rules. 
We can set the length, restrict individual words (e.g. ``3rd word must be a noun''), require that it triggers some rule but not other. 
This functionality can aid the grammar writer in the process, to see if they have missed a case or defined the tagset correctly.






