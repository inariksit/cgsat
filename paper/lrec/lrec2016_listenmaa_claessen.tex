%
% File nodalida2015.tex
%
% Contact beata.megyesi@lingfil.uu.se
%
% Based on the instruction file for EACL 2014
% which in turn was based on the instruction files for previous 
% ACL and EACL conferences.

\documentclass[11pt]{article}
\usepackage{nodalida2015}
\usepackage{times}
\usepackage{mathptmx}
\usepackage{fixltx2e}
\usepackage{url}
\usepackage{multirow}
\usepackage{latexsym}
\usepackage{cite}
\usepackage{graphicx}
%\usepackage{slantsc}
\usepackage[colorlinks=true,citecolor=blue,urlcolor=blue]{hyperref}
\usepackage{authordate1-4}
\usepackage{multirow}
%\special{papersize=210mm,297mm} % to avoid having to use "-t a4" with dvips 
%\setlength\titlebox{6.5cm}  % You can expand the title box if you really have to

\usepackage{fontspec}
\usepackage{wasysym}

\newcommand{\mainfontfamily}{Palatino}
\setmainfont[
  Ligatures=TeX,
  Scale=0.9,
]{\mainfontfamily}

\newfontfamily\scshape[Letters=SmallCaps, Numbers=Uppercase]{Palatino
  Small Caps}

\newcommand{\monofontfamily}{Monaco} % Monaco
\setmonofont[Scale=0.8]{\monofontfamily}


\usepackage{color}
\newcommand{\todo}[1]{{\color{cyan}\textbf{[TODO: }#1\textbf{]}}}

\title{Analysing Constraint Grammars with a SAT-solver}

\author{Inari Listenmaa \and Koen Claessen \\
 Chalmers University of Technology, Gothenburg, Sweden \\
 {\tt \{inari,koen\}@chalmers.se} }

\date{\today}

\begin{document}
\maketitle

\begin{abstract}
%\small
We describe a method for
 analysing Constraint Grammars by encoding the
rules in SAT.
Our tools can detect internal conflicts or redundancies in a grammar,
as well as generate examples to demonstrate the effect of some rule or
rule set.
This can help users to diagnose and improve their grammars.
No corpus is required, only a morphological lexicon.
%We test our tools with grammars from 100-1000 rules.
\end{abstract}

% Introduction & problem description
\section{Introduction}
\label{sec:intro}

Constraint Grammar (CG, \cite{karlsson1995constraint})
% \citep[CG][]{karlsson1995constraint} 
is a formalism used to disambiguate morphologically analysed text. 
A grammar consists of rules that target specific analyses for selection or removal, based on contextual tests. For example, the following rule
\begin{itemize}
\item[] \texttt{REMOVE verb IF (-1 det) ;}
\end{itemize}
removes a verb reading from a word which is preceded by a determiner.
Given the following text,
\begin{itemize}
\item[] 
\begin{verbatim}
"<the>"
        "the" det def
"<bear>"
        "bear" noun sg
        "bear" verb pl
\end{verbatim}
\end{itemize}
the rule will match to the word \emph{bear}, and remove the analysis \texttt{verb pl}.
However, if the target word has only one remaining analysis, then the rule will not apply, even if the condition is met.

CGs are valuable resources for rule-based NLP, especially for lesser
resourced languages. They are robust and can be written without large
corpora---only morphological analysis is needed. The formalism is
lightweight and language-independent, and resources can be shared
between related languages \cite{bick2006spanish}.
%,lene_trond_linda2010}.
%Mature CGs contain some thousands of rules, but even small CGs are shown to be effective \cite{lene_trond2011}.

As CGs grow larger, it gets harder for the grammar writers and users
to keep track of all the rules and their interaction.
With our tools, the users can diagnose and improve their grammars.
Some examples of conflicts are listed below:
\begin{itemize}
\item If a rule appears twice, the second occurrence will be disabled by the first
\item $R$ selects something in a context, $R'$ removes~it
\item $R$ removes something from the context of $R'$, so $R'$ can never
  apply
\item $R$ has an internal conflict, such as non-existent
tag combination, or contradicting requirements for a context word
\end{itemize}
$R$ can also be a set of rules: for instance, if one rule removes a verb in
context $C$, and another in context $\neg C$, together these rules
remove a verb in all possible cases, disabling any future rule that
targets verbs.

%In addition to analysing the whole grammar,
Our method can be used to
test individual rules. For instance, we can ask the tool to
generate example sequences that trigger rule(s) $R$ but not rule(s) $R'$. 
This can help the grammar writer to see
if they have implemented the rules in the intended way.

% They can ask specific questions about a given grammar, such as:
% \begin{itemize}
% \item Are there rules that contradict each other?
% \item Are there rules that will never fire?
% \item Generate a sequence that triggers rule(s) $R$ but not rule(s) $R'$
% %\item Generate a sequence that is ambiguous but doesn't trigger any rules
% \end{itemize}
% Our technique requires an existing morphological lexicon, compatible
% with the tag set used in the grammar, but no corpus. 
% The lexicon is needed in order to limit the possible tag combinations.
% Constraints for the structure of the sequences come from the CG rules themselves.
% We encode both of these constraints as a satisfiability problem, and use a SAT-solver to generate answers to the previous questions.

%The paper is structured as follows. Section~\ref{sec:prev} relates our work to the previous research. 
% Section~\ref{sec:implementation} discusses the implementation, 
% Section~\ref{sec:eval} presents preliminary results and discusses future work. 
%Section~\ref{sec:conclusion} concludes the paper.

%\todo{Relate to conf themes: "Methodologies and tools for LRs construction and annotation" and
%"Validation and quality assurance of LRs"}






\section{Related work}
\label{sec:prev}

% \begin{quote}Another desirable facility in the grammar development environment would
% be a mechanism for identifying pairs of constraints that contradict each
% other.
% --Atro Voutilainen, 2004
% \end{quote}

%This section describes prior research and relates our contribution to the existing approaches.
We combine elements from the following aspects of CG research:

\begin{itemize}
\item Corpus-based methods in manual grammar development \cite{voutilainen2004}
\item Optimising hand-written CGs~\cite{bick2013tuning}
\item Encoding CG in logic \cite{lager98,lager_nivre01,listenmaa_claessen2015}
\end{itemize}

In addition, there is a large body of research on automatically inducing rules, e.g. \cite{inducing_cg1996,lindberg_eineborg98ilp}.
%, \cite{lager01transformation}, \cite{asfrent14} 
However, since our work is aimed to aid the process of hand-crafting rules, we omit those works from our discussion.


\paragraph{Corpus-based methods in manual grammar development}

Hand-annotated corpora are commonly used in the development of CGs, because they give immediate feedback whether a new rule increases or decreases accuracy \cite{voutilainen2004}.
This helps the grammar writer to arrange the rules in appropriate sections, with safest and most effective rules coming first.
However, this method will not notice a missed opportunity or a grammar-internal conflict, nor suggest ways to improve.

% -- uncomment for full version ?

%\cite{voutilainen2004} gives a detailed account about best practices of grammar writing and efficient use of corpora to aid the grammar development.
%For a language with no free or tagset-compatible corpus available, \cite{tyers_reynolds2015} describe a method where they apply their rules to unannotated Wikipedia texts and pick 100 examples at random for manual check.

% CG rules are usually arranged in sections, and run in the following manner. 
% First apply rules from section 1, and repeat until nothing changes in the text. Then apply rules from sections 1--2, then 1--3 and so on, until the set includes all rules.
% The best strategy is to place the safest and most effective rules in the first sections,
% so that they make way for the following, more heuristic and less safe rules to act on.

% A representative corpus is arguably the best way to get concrete numbers---how many times a rule applied and how often it was correct---and to arrange the rules in sections based on that feedback.
% However, this method will not notice a missed opportunity or a grammar-internal conflict, nor suggest ways to improve.

% \cite{voutilainen2004} state that the around 200 rules are probably enough to resolve 50--75 \% of ambiguities in the corpus used in the development. 



\paragraph{Automatic optimisation of hand-written grammars }
% The corpus-based method can tell the effect of each single rule at their place in the rule sequence, and leaves the grammar writer to make changes in the grammar.

\cite{bick2013tuning} modifies the grammar automatically, by trying
out different rule orders and altering the contexts of the rules.
Bick reports error reduction of 7--15\% compared to the original grammars.
As a downside, the grammar writer will likely not know why exactly does the tuned grammar perform better.
% At a certain point, the grammar gets so big that it is hard to keep track of all the rules and their interactions. \cite{bick2013tuning} tries out combinations of moving rules in different sections or removing them in total, and in parallel, making their contexts stricter or less strict. 
% This is a valuable tool, especially for grammars that are so big that it's hard to keep track of. Program can try all combinations whereas trying to make sense out of a huge set of rules would be hard for humans.



\paragraph{CG encoded in logic}

%\cite{lager98} presents a CG-like shallow parsing system encoded in logic, and \cite{lager_nivre01} continues with a reconstruction of four different formalisms.
%The earlier works on logical reconstruction don't envision grammar analysis as one of the use cases,
\cite{lager98} and \cite{lager_nivre01} reconstruct the CG formalism in first-order predicate logic.
%, and \cite{listenmaa_claessen2015} implement a CG compiler using a SAT-solver.
Grammar analysis is a natural use case, due to some key features of the logical reconstruction.
The traditional CG compiler 
%, such as VISL CG-3,
 cannot capture any dependencies between rules.
% It discards analyses immediately: the output of the $i^{th}$ rule becomes the input of the $i+1^{th}$ rule, but there is no information which rules have been applied before.
In contrast, a logic-based CG compiler does that by default. 
The rules are modelled as implications and composed in the order of the rule sequence, such that 
the consequent from the $i^{th}$ rule becomes the antecedent of the $i+1^{th}$ rule.
Given this design, we added on top a way to ask for solutions with certain properties.
%, such as ``after applying rules $0-i$, is it possible for $i+1$ to apply''.

% The key elements in the ontology of CG are positions, words and sets of tags.
% Rule order is denoted by predicate $pos^i$(word, [tag]), 
% which denotes the part of speech of a given word after applying the $i^{th}$ rule.
% The rules are modelled as implications, of the form below:






% * Describe problem: CGs are huge & prone to mistakes,
%  * we are looking at conflicts such as ...
%  * if you forgot a case
%  * some evidence that big CGs have conflicts and this is a real problem
%  * Eckhard's paper to explain why conflicting rules are a problem
%  * Goal: help grammar writer while they are writing the grammar, to avoid these kinds of problems
%  * Examples

% * Describe the technique







% Implementation details; very brief for the abstract
\section{Implementation}
\label{sec:implementation}

In this section, we describe the implementation of the tool.
The SAT-encoding we use is similar to the one introduced in \cite{listenmaa_claessen2015}, with one key difference: in this paper, we operate on {\em symbolic sentences} instead of concrete sentences from a corpus. The idea is that the SAT-solver is going to find the concrete sentence for us.

\paragraph{Preliminaries}

Our analysis operates on one rule $R$, and is concerned with answering the following question: ``Does there exist an input sentence $S$ that can trigger rule $R$, even after passing all rules $R'$ that came before $R$?''

Before we can do any analysis any of the rules, we need to find out what the set of all possible readings of a word is. We can do this by extracting this information from a lexicon, but there are other ways too. In our experiments, the number of readings has ranged from about 300 to about 6000. 

Furthermore, when we analyse a rule $R$, we need to decide the {\em width} $w(R)$ of the rule $R$: How many different words should there be in a sentence that can trigger $R$? Most often, $w(R)$ can be easily determined by looking at how far away the rule context indexes in the sentence relative to the target. For example, in the rule mentioned in the introduction, the width is 2.

If the context contains a \verb!*!, we may need to make an approximation of $w(R)$ which may result in false negatives later on in the analysis.

\paragraph{Symbolic sentences}

We start each analysis by creating a so-called {\em symbolic sentence}, which is our representation of the sentence $S$ we are looking for. A symbolic sentence is a sequence of {\em symbolic words}; a symbolic word is a table of all possible readings that a word can have, where each reading is paired up with a SAT-variable.

The number of words in the symbolic sentence we create when we analyse a rule $R$ is $w(R)$. For the rule in the introduction, we have $w(R)=2$ and a symbolic sentence may look as follows:
\begin{center}
\begin{tabular}{c|c|c}
word1 & word2 & reading \\
\hline
$v_1$ & $w_1$ & det def \\
$v_2$ & $w_2$ & noun sg \\
$v_3$ & $w_3$ & noun pl \\
$v_4$ & $w_4$ & verb sg \\
$v_5$ & $w_5$ & verb pl \\
\end{tabular}
\end{center}
Here, $v_i$ and $w_j$ are SAT-variables belonging to word1 and word2, respectively. We can also see that the possible number of readings here was 5.

The SAT-solver contains extra constraints about the variables. Input sentences should have at least one reading per word, so we add the following two constraints:
\begin{center}
\begin{tabular}{c}
$v_1 \vee v_2 \vee v_3 \vee v_4 \vee v_5$, \\
$w_1 \vee w_2 \vee w_3 \vee w_4 \vee w_5$ \\
\end{tabular}
\end{center}
Any solution to the constraints found by the SAT-solver can be interpreted as a concrete sentence with $w(R)$ words that each have a set of readings.

\paragraph{Applying a rule}

Next, we need to be able to apply a given rule $R'$ to a symbolic sentence, resulting in a new symbolic sentence.

For example, if we apply the rule from the introduction to the symbolic sentence above, the result is the following symbolic sentence:
\begin{center}
\begin{tabular}{c|c|c}
word1 & word2 & reading \\
\hline
$v_1$ & $w_1$ & det def \\
$v_2$ & $w_2$ & noun sg \\
$v_3$ & $w_3$ & noun pl \\
$v_4$ & $w_4'$ & verb sg \\
$v_5$ & $w_5'$ & verb pl \\
\end{tabular}
\end{center}
The example rule can only affect readings of word2 that have a ``verb'' tag, so we create only two new variables $w_4'$ and $w_5'$ for the result, and reuse the other variables. We add the following constraint for $w_4'$ is:
\begin{center}
\begin{tabular}{c}
$w_4' \Leftrightarrow [ w_4 \wedge \neg{}(v_1 \wedge (w_1 \vee w_2 \vee w_3)) ]$ \\
\end{tabular}
\end{center}
In other words, after applying the rule, the reading ``verb sg'' (represented by the variable $w_4'$) can only be in the resulting sentence exactly when (1) ``verb sg'' was a reading of the input sentence (so $w_4$ is true) and (2) the rule has not been triggered (the rule triggers when $v_1$ is true and at least one of the non-verb readings $w_1 \dots w_3$ is true). We add a similar constraint for the new variable $w_5'$:
\begin{center}
\begin{tabular}{c}
$w_5' \Leftrightarrow [ w_5 \wedge \neg{}(v_1 \wedge (w_1 \vee w_2 \vee w_3)) ]$ \\
\end{tabular}
\end{center}

\paragraph{Putting it all together}

Once we know how to apply one rule $R'$ to a symbolic sentence, we can apply all rules preceding the rule $R$ that is under analysis. We simply apply each rule to the result of applying the previous rule. In this way, we end up with a symbolic sentence that represents all sentences that could be the result of applying all those rules.

Finally, we can take a look at the rule $R$ we want to analyse. Here is an example:
\begin{itemize}
\item[] \texttt{REMOVE det IF (1 verb) ;}
\end{itemize}
If we take the symbolic sentence above as input, we want to ask wether or not it can trigger the rule $R$. We do this by adding some more constraints to the SAT-solver.

First, the context of the rule should be applicable, meaning that the second word should have a reading with a ``verb'' tag:
\begin{center}
\begin{tabular}{c}
$w_4' \vee w_5'$
\end{tabular}
\end{center}
Second, the rule should be able to remove the ``det'' tag, meaning that the first word should have a reading with a ``det'' tag, and there should be at least one other reading:
\begin{center}
\begin{tabular}{c}
$v_1 \wedge (v_2 \vee v_3 \vee v_4 \vee v_5)$
\end{tabular}
\end{center}
If the SAT-solver can find a solution to all constraints generated so far, we have found a concrete sentence that satisfies our goal. If the SAT-solver cannot find a solution, it means that there are no sentences that can ever trigger rule $R$. This means that there is something wrong with the grammar.

% \paragraph{Other questions we can ask}

% Apart from finding out what rules prevent others from applying, we can also find out if there is a conflict that is rule-internal, such as nonexisting tag set or contradicting requirements of a context word (e.g. a word must be unambiguously two different POS).

% In addition to analysing a whole grammar, 
% we can construct all kinds of symbolic sentences to test out individual rules. 
% We can set the length, restrict individual words (e.g. ``3rd word must be a noun''), require that it triggers some rule but not other. 
% This functionality can aid the grammar writer in the process, to see if they have missed a case or defined the tagset correctly.



% Preliminary evaluation and future work
\section{Evaluation}
\label{sec:eval}

We tested three grammars to find conflicting rules: 
Dutch\footnote{\url{https://svn.code.sf.net/p/apertium/svn/languages/apertium-nld/apertium-nld.nld.rlx}},
with 59 rules; 
Spanish\footnote{\url{https://svn.code.sf.net/p/apertium/svn/languages/apertium-spa/apertium-spa.spa.rlx}},
with 279 rules; and 
Finnish\footnote{\url{https://github.com/flammie/omorfi/blob/master/src/vislcg3/omorfi.cg3}},
with 1185 rules. We left out \textsc{add}, \textsc{map} and other rule
types introduced in CG-3, and only tested \textsc{remove} and \textsc{select} rules.
The results for Dutch and Spanish are shown in table~\ref{table:res},
and the results for Finnish in table~\ref{table:resFin}.

A natural follow-up evaluation would be to compare the performance of the
grammar in the original state, and after removing the conflicts found
by our tool. Unfortunately, we were not able to perform such
evaluation, due to the lack of available gold standard corpora for all
the languages.


\begin{table}[]
\centering
\begin{tabular}{|l|l|l|l|}

\hline
                   & \textsc{nld}  & \textsc{spa}lem  & \textsc{spa}nolem \\ \hline
\# rules           & 59            & 279       & 279     \\ \hline
\# readings        & 336           & 3905      & 1735    \\ \hline
\# conflicts       & 7             & n         & m    \\ \hline
\clock{} all rules & ??s           & + 1h      & 30m-1h   \\ \hline
% \clock{} last rule & ??s           & ??s       & ?min ??s    \\ \hline
\clock{} no amb. 
classes            & 10ish sec       & ??min ??s    & ?h ??min    \\ \hline

%\# variables: last rule & 33915         & 247153            &    \\ \hline

\end{tabular}
\caption{\todo{make more tables}}
\label{table:res}
\end{table}

\begin{table}[]
\centering
\begin{tabular}{|l|l|l|l|}

\hline
              & n clitics & m clitics & only underspecified \\ \hline
\# rules      & 1185      & 1185      & 1185\\ \hline
\# readings   & many & more & only like <2000   \\ \hline
\# conflicts  & n   & m        & super many    \\ \hline
\clock{} all rules       & ??s              & ??min ??s    & ?h ??min    \\ \hline
\clock{} last rule       & ??s              & ??s          & ?min ??s    \\ \hline


\end{tabular}
\caption{Special table for Finnish}
\label{table:resFin}
\end{table}


The experiments revealed problems in all grammars. For the smaller
grammars, we were able to verify manually that the detected rules were
true positives.
%The results for the Finnish grammar are inconclusive, 
% For the Finnish grammar, we did not have time to
% investigate all of them, but among those we could, we found both true and false negatives.
We did not check for false negatives in any of the grammars. However, the test reported a high number of errors even in the smallest grammars, and some of them are intuitively not that bad; we 


\paragraph{Dutch} The Dutch grammar had two kinds of errors: rule-internal, and rule interaction. As for rule-internal conflicts, one was due to a misspelling in the list definition for personal pronouns, which rendered 5 rules ineffective. The other was about subreadings: the genitive \emph{s} is analysed as a subreading in the Apertium morphological analyser, but it appeared in 2 rules as the main reading. 

There was one genuine conflict with rule interaction, shown below:

\begin{itemize}
\item[] 
\begin{verbatim}REMOVE Adv IF (1 N) ;
REMOVE Adv IF (-1 Det) (0 Adj) (1 N) ;
\end{verbatim}
\end{itemize}

Two rules, which both remove an adverb, were in an order where the first has more broad condition: remove adverb if followed by a noun. In contrast, the second rule has a stricter requirement: only remove adverb if it is preceded by a determiner; the adverb itself is ambiguous with an adjective, and followed by a noun. The proble is that the first rule removes the adverb in all possible cases, and the second will not have any chance to act. If the rules were in the opposite order, then there would be no problem.


%%% Nevermind, this wasn't a conflict, my program was just buggy :-P
%This is not a conflict, just pointing out an example how the SAT-solver constructs the only possible sentence where this works.

%\begin{itemize}
%\item[] 
%\begin{verbatim}
%SELECT DetPosNotZijn IF (1 Noun) ;
%SELECT DetPos IF (-1 (vbser pres p3 sg)) 
%                 (0 "zijn") (1 Noun);
%\end{verbatim}
%\end{itemize}

% SELECT DetPosNotZijn IF (1 Noun);
%If the first rule fires, then the condition of the second rule cannot be true: there can never be a ``zijn'' in the target, because the previous rule has selected everything that is a possessive determiner and not ``zijn''. 
%So, the first rule must not fire. It cannot be because condition does not hold: the condition (1 Noun) is shared with the second rule, and  

We also tested rules individually, in a way that a grammar writer might use our tool when writing new rules.
The following rule was one of them:

\begin{itemize}
\item[] 
\texttt{SELECT DetPos IF (-1 (vbser pres p3 sg)) (0 "zijn") (1 Noun);}
\end{itemize} 

As per VISL CG-3, the condition \texttt{(0 "zijn")} does not require
that the input would have determiner and ``zijn'' in the same reading:
just that there is a reading with any determiner, and a reading with any ``zijn'' in the cohort. 
Can we catch this imprecise formulation with our tool? The example word constructed by the SAT-solver may as likely be one of the following:

\begin{itemize}
\item[a.] \begin{verbatim}
"w2"
    w2<det><pos><f><sg>
    w2<vbser><inf><"zijn">
\end{verbatim}

\item[b.] \begin{verbatim}
"w2"
    w2<det><pos><mfn><pl><"zijn">
\end{verbatim}
\end{itemize}

Incidentally, the first example of a rule conflict had a similar
requirement, but used correctly: the intended semantics was indeed
that the adjective and the adverb are in different readings. However,
in this case, the precise way to express the target is \texttt{SELECT DetPos + "zijn"}.

% In the latter case, the grammar writer would hopefully notice the imprecise definition and change their definition. However, this is more of a happy side effect than intended feature---the grammar writer cannot count on our tool to notice all similar cases.

\paragraph{Spanish} The Spanish grammar had proportionately the highest number of errors. \todo{confirm; I assume this is gonna be so, but let's see Fin and Por}.  The grammar we ran is like the one found in the Apertium repository, apart from 2 changes: we fixed some typos (capital O for 0) in order to make it compile, and commented out two rules that used regular expressions, because our solution does not handle them properly (see more on Future work/Implementation).

Again, we can classify the conflicts into internal and interaction. As an example of internal conflict, there are \todo{n} rules that use \texttt{SET Cog = (np cog)}---the problem is that the tag \texttt{cog} does not exist in the Apertium dictionary for Spanish. It is likely that this grammar has been written for an earlier version, where such tag has been in place.




The Spanish grammar has a high number of interaction conflicts, \todo{n} with the same pattern:

\begin{itemize}
\item[] 
\begin{verbatim}REMOVE (a) IF (-1 x) (1 y) ;
REMOVE (a) IF (-1 x) (1 y) (NOT -1 z);
\end{verbatim}
\end{itemize}


This is another instance of more general rule applying before more
specific. Negated context works as well to make a rule stricter. The
second rule will never apply, because the first will catch all
cases. For a full list of found conflicts, see \todo{\url{actual://link.here/spanish/error.log}}

In addition, there was a number of definitions that were never
used. The program doesn't currently point them out to the user, but
that is a trivial addition; just like rule-internal conflicts, it does
not require any semantic analysis. 
We also discovered problems in some set definitions, when we tried to
use the set definitions from the grammar directly as our readings.
For instance, the following definition requires the word to be all of
the listed parts of speech at the same time:
\begin{itemize}
\item[] 
\texttt{SET NP\_Member = N + A + Det + PreAdv + Adv + Pron ;}
\end{itemize}

This particular definition was never used in the rules, so it would not
have caused a conflict. We noticed it by accident, when the SAT-solver
offered it in an example sequence that, meant for another rule.


As with the Dutch grammar, we ran the tool on individual rules and
examined the sequences that were generated. None of the following was
marked as a conflict, but looking at the output indicated that there
are multiple interpretations, such as whether two analyses for a
context word should be in the same reading or different readings.
We observed also cases where the grammar writer has specified desired
behaviour in comments, but the rule does not do what the grammar
writer intended. 


% \begin{itemize}
% \item[] 
% \begin{verbatim}
% REMOVE Vblex IF (0 "como") (*-1 Vblex BARRIER Cnj_Rel) ;
% \end{verbatim}
% \end{itemize}

% Based on the comments in the file, the grammar writer wants to
% disambiguate the word form ``como'', which can be either a
% preposition, adverb or the first person singular of the verb
% ``comer''. 
% However, the rule here addresses the \emph{lemma} ``como'' instead of the
% word form, hence the SAT-solver cannot create a solution where
% \texttt{vblex} and ``como'' as a lemma are in the same
% reading. 
% A precise definition for this rule would target \texttt{vblex "<como>"} 
% or \texttt{vblex "comer"}, so that it becomes obvious that those two
% are supposed to be in the same reading. 

% Currently, as all the word forms are allowed to combine with anything,
% the ambiguity class constraint does not point this
% The automated 0-to-target conversion could come in handy here too.
% However, if the 

% TODO An example of the same but in condition. that the ambiguity class constraint, when
% implemented properly, should find out.



\begin{itemize}
\item[] 
\texttt{REMOVE Podar  IF (0 Podar) (1 Adv) (2C Inf OR Enc) ;} \\
\texttt{REMOVE Sentar IF (0 Sentar) (1C Inf OR Enc) ;}
\end{itemize}

The comments make it clear that the first rule is meant to disambiguate between poder and podar, and the second rule between sentir and sentar. But the rule does not mention anything about ``poder'' nor ``sentir''; the SAT-solver gave a triggering sequence without 

This is actually a shortcoming of our way of handling the lemmas. If the lemmas and word forms were integrated to the analyses in a more feasible way, then the ambiguity class constraints would create a sequence where ``podar'' was ambiguous with ``poder'', instead of randomly chosen readings. So, even though this was not marked as a conflict, we can say that the weird output of the tool was unnecessary, and by better implementation, the tool would have created more realistic example sequence.

We found the same kind of formulation in many other rules and grammars.
To catch them more systematically, we could add a feature that alerts in all cases where a condition with 0 is used, and automatically construct a version of the rule that moves whatever tags in the 0-condition into the target, then asks the user which one was meant.




\paragraph{Finnish} 

\todo{Rerun the tests and rewrite this whole section!} 

The results for the Finnish grammar are shown separately, in table~\ref{table:resFin}.
We were not able to use the method with ambiguity classes at all---expanding the Finnish morphological lexicon results in \todo{100s of gigabytes?} of word forms, which is simply too big for our method to work. 
For future development, we will see if it is possible to manipulate the finite automata directly, instead of relying on the output in text.

In addition to the previous problem, we had a large number of readings. In order to make the test run faster, we tried a few shortcuts. First, we restricted the amount of clitics per word form, and even ignored all the specified readings

The Finnish grammar was the largest and most complex of all the
grammars we have tested. 





the grammar was originally written in 1995, and converted to the
Omorfi tagset by \cite{pirinen2015}, but there were 15 rules that used the old convention for possessive suffixes.


To improve performance, we ignored the fully specified readings from the
lexicon and used only tag combinations from the grammar:
for example, the rule \texttt{REMOVE (verb sg) IF (-1 det)}
gives us ``verb sg'' and ``det''.
The shortcut works most of the time, but it is possible to
miss some cases: e.g. \texttt{SELECT PRON + REL IF (0 NOM)} 
assumes ``pron rel nom'', but our method only gives
``pron rel'' and ``nom'' separately. 
17 false positives were caused by this shortcut. 
In addition, 2 false positives were caused by the naive implementation of
rules with \verb!*! in their context.

Due to these reasons, the results about the Finnish grammar are
inconclusive. We only include them to show some preliminary data on larger grammars.
%For a condition situated $n$ or more words to the left or right, we only create a word exactly $n$ words away. This is something to be fixed in the future.

\paragraph{Performance} The speed of the check is an obvious issue,
especially for the Finnish grammar. 
As mentioned before, we used 2000 readings extracted from the grammar.
With the full set of 6000 readings from the lexicon, the speed
was even worse: checking the last rule took 30 minutes, as opposed to
3 minutes with 2000 readings. 
%We never finished running the whole rule set with 6000 readings, but
%it would have taken potentially days.
We believe we can still signifcantly improve the performance of the SAT-encoding. Currently it is just a naive proof-of-concept implementation, and we have not tried to optimise for speed at all.

% A Finnish symbolic word has around 5700 variables in the beginning of the analysis, each representing a possible tag combination.
% Around 3700 come from the Apertium morphological lexicon, excluding clitics, and around 2000 are defined in the grammar, as a target (\texttt{REMOVE ("bear" verb) IF ...}) or as a condition (\texttt{REMOVE ... IF (-1 ("bear" noun))}. The majority of them are lexical, whereas we don't include anything but morphological tags from the lexicon.
% For Spanish and Dutch, there are around 2000 and 300 tag combinations respectively.
% The Spanish lexicon includes up to two clitics (e.g. \emph{dámelo} `give.me.it') in the same word form, but Dutch has none in the lexicon.

% The SAT problem for checking one rule grows depending on how many rules come before it, how long the context is, and how many tag combinations are there. The running time of the last rule gives some indication ...


% Not all questions are so expensive.
% For instance, checking the tag sets in the Finnish grammar doesn't require the context of the other rules at all,
%  The majority of the correctly detected errors in the Finnish grammar were rule-internal---these errors don't need the context of the other rules at all.


\section{Future work and conclusions}
\label{sec:conclusion}

The evaluation indicates that the tool finds cases that are not
obvious to the human eye.

One additional feature is to suggest reformattings for a rule. Recall
figure~\ref{fig:regroup} from the introduction; in that case, the
original rule was written by the original author, and another
grammarian thought that the latter form is nicer to read. Doing the
reverse operation could also be possible. If a rule with long
disjunctions conflicts, it may be useful to split it into smaller
conditions, and eliminate one at a time, in order to find the
reason(s) for the conflict.



Our solution to hardcode the tag combinations in the readings is feasible for simple morphology, but it can cause problems with more complex morphology. One big downside is that in order to implement \textsc{ADD}, \textsc{ADDREADING} and \textsc{MAP}, we need to be prepared for new readings--even if the lexicon gives all readings that exist in the lexicon, the user might give a nonexistent reading, or in the case of MAP, a syntactic tag, which is (by definition) not in the lexicon. A more scalable solution would be to make each tag a variable, and ask the question ``can this reading be a noun? how about singular? how about conditional?'' separately for each tag. Then we could lift the restriction of tag combinations into a SAT solver: make SAT clauses that prohibit a comparative to go with a verb, or conditional with a noun.
Another benefit of this solution is the addition of lemmas and word forms, as well as regular expressions: currently, if we want to add one more lemma for a verb, we need to create as many new variables as there are distinct verb forms---easily hundreds for languages with rich morphology.
\todo{less rambling: For rules with regular expressions, this would blow up even more: there are rules that only address whether the word form in a condition starts with a lowercase letter, or whether it ends in a certain suffix. A realistic, albeit unsatisfactory option is just to treat it as an underspecified reading, and offer that same regular expression to the user when generating a symbolic sentence. Another, less feasible, solution is to grep all the word forms or lemmas that match the regular expression in question---in the worst case (e.g. lower case), this will match all the entries in the expanded lexicon.}


As for longer-term goals, we want to handle the full expressivity of CG-3,
with \textsc{map}, \textsc{add} and \textsc{substitute} rules, and
dependency structure. This also means finding different kinds of conflicts.
As soon as the tools are mature enough, we want to
evaluate them with actual grammar writers,
in comparison with a corpus-based method or machine learning.
Finally, if the method proves feasible for CG, we want
to try applying it to other grammar formalisms.


% * Preliminary results
%  - dutch & spanish
%  - mention scalability
%  - talk about size of SAT problem -- give number of SAT clauses for the last rule in the biggest grammar I have

% * Future work
%  - analysing different grammar formalisms
%  - asking different questions
%  - restrict yourself to readings that are actually words


\section*{Acknowledgments}
We thank Eckhard Bick for the idea to apply SAT to CG analysis. 


\bibliographystyle{acl}
\bibliography{cg}

\end{document}