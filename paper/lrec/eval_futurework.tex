\section{Evaluation and Future Work}
\label{sec:eval}

We tested three grammars to find rules that cannot apply. 
The smallest grammar was
Dutch\footnote{\url{https://svn.code.sf.net/p/apertium/svn/languages/apertium-nld/apertium-nld.nld.rlx}},
with 59 rules; second was
Spanish\footnote{\url{https://svn.code.sf.net/p/apertium/svn/languages/apertium-spa/apertium-spa.spa.rlx}},
with 279 rules, and the largest was
Finnish\footnote{\url{https://github.com/flammie/omorfi/blob/master/src/vislcg3/omorfi.cg3}},
with 1185 rules. We left out \textsc{add}, \textsc{map} and other rule
types introduced in CG-3, and only tested the traditional
\textsc{remove} and \textsc{select} rules.
The results are shown in table~\ref{table:res}. 


\begin{table}[]
\centering
\begin{tabular}{|l|l|l|l|}
\hline
                      & \textsc{nld}  & \textsc{spa}  & \textsc{fin}  \\ \hline
\# rules      & 59              & 279      & 1185     \\ \hline
\# readings      & 276             & 2191     & 2012    \\ \hline
\# conflicts       & 1               & 1        & \todo{n}    \\ \hline
\clock{} all rules       & 7s              & 3min 22s    & 3h 26min    \\ \hline
\clock{} last rule       & 4s              & 7s                & 3min 40s    \\ \hline
% This is with just 2000ish variables for Finnish, below is with 5700

% \clock{} last rule       & 4s              & 7s          & 30min 25s    \\ \hline
% \clock{} longest rule    & ?s              & ?s                & ?min     \\ \hline
%\# variables: last rule & 33915         & 247153            &    \\ \hline

\end{tabular}
\caption{Detecting rules that cannot be applied}
\label{table:res}
\end{table}

\paragraph{Correctness} 
The test revealed problems in all grammars, and for the smaller
grammars, we were able to verify manually that the detected rules were
true negatives. However, for the Finnish grammar, we did not have time
to investigate all of them; in addition, we suspect that some of
them are false negatives.
We did not check for false positives in any of the grammars.

The Spanish grammar had an erroneous set definition, where a word was
required to have almost all POS tags at the same time. After fixing
the set definition, our test did not reveal other problems.
The Dutch grammar had the following rules in the given order, making the latter impossible to apply:
\begin{itemize}
\item[] 
\begin{verbatim}REMOVE Adv IF (1 Noun) ;
REMOVE Adv IF (-1 Det) (0 Adj) (1 Noun) ;
\end{verbatim}
\end{itemize}

For the Finnish grammar, we could verify a number of tagset errors. 
The grammar was originally written in 1995, and converted to the
Omorfi tagset by \cite{pirinen2015}. For example, the Omorfi tagset
marks a possessive suffix with \texttt{Px}, e.g. \texttt{PxPl2} for second person plural, but the
grammar used the old convention \texttt{PL2} in the rules.
As a result, all rules with possessive suffixes were detected as
impossible.

Because of performance issues, we ignored the readings from the
lexicon and used only tag combinations that were named in the grammar.
For example, the rule \texttt{REMOVE (verb sg) IF (-1 det)}
gives us ``verb sg'' and ``det'', as opposed to fully specified
readings in the lexicon.
The shortcut will work most of the time, but it is possible to
miss some cases: e.g. \texttt{SELECT (verb sg) IF (0 p3)} 
requires a full ``verb sg p3'' reading, but our method only gives
``verb sg'' and ``p3'' separately. 
Some of the false negatives seem to come from this shortcut, or our naive
implementation of rules with \verb!*! in their context.
Due to these reasons, we cannot give conclusions about the Finnish
grammar; we include the results here only to have some preliminary
data on scaling up to larger grammars.
%For a condition situated $n$ or more words to the left or right, we only create a word exactly $n$ words away. This is something to be fixed in the future.

\paragraph{Running time} The speed of the check is an obvious issue,
especially for the Finnish grammar. 
As mentioned before, we took a shortcut and used 2000 readings from
the grammar.
With the full set of 6000 readings from the lexicon, the speed
was even worse: checking the last rule took 30 minutes, as opposed to
3 minutes with 2000 readings. 
%We never finished running the whole rule set with 6000 readings, but it would have taken potentially days.
However, the code is currently just a proof of concept, and we plan to
make an effort in optimisation.


% A Finnish symbolic word has around 5700 variables in the beginning of the analysis, each representing a possible tag combination.
% Around 3700 come from the Apertium morphological lexicon, excluding clitics, and around 2000 are defined in the grammar, as a target (\texttt{REMOVE ("bear" verb) IF ...}) or as a condition (\texttt{REMOVE ... IF (-1 ("bear" noun))}. The majority of them are lexical, whereas we don't include anything but morphological tags from the lexicon.
% For Spanish and Dutch, there are around 2000 and 300 tag combinations respectively.
% The Spanish lexicon includes up to two clitics (e.g. \emph{dámelo} `give.me.it') in the same word form, but Dutch has none in the lexicon.

% The SAT problem for checking one rule grows depending on how many rules come before it, how long the context is, and how many tag combinations are there. The running time of the last rule gives some indication ...


% Not all questions are so expensive.
% For instance, checking the tag sets in the Finnish grammar doesn't require the context of the other rules at all,
%  The majority of the correctly detected errors in the Finnish grammar were rule-internal---these errors don't need the context of the other rules at all.


\paragraph{Future work} 

We will work on the performance, and fix any errors we found in the
evaluation, such as the issue with non-exact position.
Another obvious improvement is to restrict possible ambiguities
in the symbolic words.
Currently we have only hard-coded some
very basic assumptions, such as ``word cannot be ambiguous between a
boundary and a real word'', but we want to extract these restrictions
automatically.
For instance, the Spanish lexicon contains the entries
\texttt{casa:casa<n><sg>}  and  \texttt{casa:casar<v><sg><p3>}, thus
we know that a word can be ambiguous between a noun and a verb.
By adding these restrictions to the symbolic words, we hope to
restrict the search space, and make the SAT problem easier to solve.

%We will also improve usability, by finding out which rules contrad
After this, we want to handle the full expressivity of CG-3, with \textsc{map}, \textsc{add} and \textsc{substitute} rules.
At that stage, it would be interesting to test our program with actual grammar writers,
in comparison with a corpus-based method or machine learning.
Finally, if the method proves feasible for Constraint Grammars, we want
to try applying it to other, more complex grammar formalisms.


% As another line of investigation, we want to detect more kinds of conflicts.
% For instance, say we have the following rules:

% \begin{itemize}
% \item[] \begin{verbatim}
% REMOVE verb IF (-1 det) ;
% REMOVE noun IF (-1 det) ;
% \end{verbatim}
% \end{itemize}

% With our symbolic sentence, these rules will be no problem; to apply the latter, we only need to construct a target that has a realistic ambiguity with a noun; the verb is gone already.
% However, in the context of real life texts, the grammar writer might actually mean the following:

% \begin{itemize}
% \item[] \begin{verbatim}
% REMOVE verb IF (-1 det) (0 noun) ;
% REMOVE noun IF (-1 det) (0 verb) ;
% \end{verbatim}
% \end{itemize}

% In that case, those rules are contradictory, and we would like to find it out.



% * Preliminary results
%  - dutch & spanish
%  - mention scalability
%  - talk about size of SAT problem -- give number of SAT clauses for the last rule in the biggest grammar I have

% * Future work
%  - analysing different grammar formalisms
%  - asking different questions
%  - restrict yourself to readings that are actually words