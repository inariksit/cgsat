\section{Evaluation and Future Work}
\label{sec:eval}

We tested three grammars to find rules that cannot apply. The smallest grammar was Dutch\footnote{\url{https://svn.code.sf.net/p/apertium/svn/languages/apertium-nld/apertium-nld.nld.rlx}}, with 59 rules; second was Spanish\footnote{\url{https://svn.code.sf.net/p/apertium/svn/languages/apertium-spa/apertium-spa.spa.rlx}}with 279 rules, and the largest was Finnish\footnote{\url{https://github.com/flammie/omorfi/blob/master/src/vislcg3/omorfi.cg3}}, with 1185 rules.
For the smaller grammars, we were able to verify manually that the results are true.

\begin{table}[]
\centering
\begin{tabular}{|l|l|l|l|}
\hline
                      & \textsc{nld}  & \textsc{spa}  & \textsc{fin}  \\ \hline
Rules in grammar      & 59              & 279               & 1185     \\ \hline
Conflicts found       & 1               & ???               & ???    \\ \hline
Time: all rules       & 7s              & 3min 22s          & 3h 26min    \\ \hline
Time: last rule       & 4s              & 7s                & 3min 40s    \\ \hline
Time: longest rule    & ?s              & ?s                & ?min     \\ \hline

\end{tabular}
\caption{Results}
\label{table:res}
\end{table}

The first row shows the whole number of rules. For the Finnish grammar, we took only \textsc{remove} and \textsc{select} rules. The other two only contained those rules to start with.

We ran for all three grammars the following thing.
For each rule, run all the rules before it to the symbolic sentence, and then ask if that rule can fire.
This test revealed problems in all grammars.
The Dutch grammar had the following rule sequence, making the latter impossible to apply:

\begin{itemize}
\item[] \texttt{REMOVE adv IF (1 n|np) ;}
\item[] \texttt{REMOVE adv IF (-1 det) (0 adj) (1 n|np) ;}
\end{itemize}

The Spanish grammar had an erroneous set definition, where a word was required to have almost all POS tags at the same time. After fixing the set definition, our test didn't reveal other problems.
The program marked __ problem rules for the Finnish grammar, but we didn't have time to verify if they were problematic or if our program is buggy.

As seen in table~\ref{table:res}, checking every rule in the Finnish grammar takes currently 3.5 hours.
A rule takes longer to run if it appears later in the rule sequence: 1000 previous rules is a larger SAT problem than 10 previous rules.
This can be seen from the time it took to compute the result for the last rule. For Finnish, the last rule took almost 4 minutes---had this been the average, the whole grammar would have taken 18 hours instead of 3.5.
We also measured the running times for the longest rules in the grammar.
\todo{find longest rule, change place of that rule, see what matters more. Promise: we can be smarter about SAT things.}



For future work, we want to improve the performance and scale up to larger grammars, and handle the full expressivity of CG-3, with \textsc{map}, \textsc{add} and \textsc{substitute} rules.
We want to detect more kinds of conflicts.
For instance, say we have the following rules:

\begin{itemize}
\item[] \texttt{REMOVE verb IF (-1C det) ;}
\item[] \texttt{REMOVE noun IF (-1C det) ;}
\end{itemize}

With our symbolic sentence, this rule set will be no problem; after all, there can be something else in the target beside noun and a verb.
However, if the grammar writer actually means the following

\begin{itemize}
\item[] \texttt{REMOVE verb IF (-1C det) (0 noun) ;}
\item[] \texttt{REMOVE noun IF (-1C det) (0 verb) ;}
\end{itemize}

then those rules are contradictory, and we would like to find it out.

\todo{Compare with a corpus-based test: do we find the same problems?}

Finally, we want to test the approach to other grammar formalisms.



% * Preliminary results
%  - dutch & spanish
%  - mention scalability
%  - talk about size of SAT problem -- give number of SAT clauses for the last rule in the biggest grammar I have

% * Future work
%  - analysing different grammar formalisms
%  - asking different questions
%  - restrict yourself to readings that are actually words