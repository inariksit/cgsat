%
% File nodalida2015.tex
%
% Contact beata.megyesi@lingfil.uu.se
%
% Based on the instruction file for EACL 2014
% which in turn was based on the instruction files for previous 
% ACL and EACL conferences.

\documentclass[11pt]{article}
\usepackage{nodalida2015}
\usepackage{times}
\usepackage{mathptmx}
%\usepackage{txfonts}
\usepackage{url}
\usepackage{latexsym}
\usepackage{listings}
\special{papersize=210mm,297mm} % to avoid having to use "-t a4" with dvips 
%\setlength\titlebox{6.5cm}  % You can expand the title box if you really have to

\title{Towards ....}

\author{Inari Listenmaa \\
  Affiliation / Address line 1 \\
  Affiliation / Address line 2 \\
  Affiliation / Address line 3 \\
  {\tt inari@chalmers.se} \\\And
  Koen Claessen \\
  Affiliation / Address line 1 \\
  Affiliation / Address line 2 \\
  Affiliation / Address line 3 \\
  {\tt koen@chalmers.se} \\}

\date{}

\begin{document}
\maketitle
\begin{abstract}
We implement Constraint Grammar using a SAT solver.

We experiment with different strategies to maximise the number of instances to apply rules;
this will also lead to different orderings of the rules and ways of solving conflicts.


\end{abstract}


\section{Introduction}
Constraint Grammar (CG) was first introduced by \cite{KarlssonTODO}. 
It is a tool for disambiguating output by morphological analyser.


\section{Related work}
\label{sect:related}

\cite{lager1998, lager2000}  CG and FSIG (Finite State Intersectional Grammar) inspired POS tagging using theorem prover.
% What we do better: implement something that takes existing CG rules and applies them to text, trying to mimic/outperform current CG implementations.

Transformation-Based Learning of Rules for CG tagging (the paper with rules where ordering didn't matter so much)



\section{Implementation using SAT}
\label{sect:pdf}

SAT solving is based on a technique called unit propagation:
a set of complex Boolean clauses is simplified, starting from unit
clauses, which consist of just a single variable, and working up to a
solution, where all variables in the set have a True or False value.

SAT solving is used for many applications where constraints must be
satisfied, such as software and hardware verification.
Application of logic to POS tagging or shallow parsing isn't new;
Lager~\shortcite{lager1998, lager2000} presents POS tagging rules as logic, and uses
a theorem prover (DisLog, TODO cite) to implement small constraint rules.
Inspired by the approach, our implementation tries to emulate the CG
formalism in behaviour and features offered.

We make each analysis a variable. For instance, the fifth line denotes
a hypothesis that the correct analysis of the 3rd item in the sentence
is \texttt{<V,Sg>}. We call this \texttt{v5}.

\begin{lstlisting}
((1,["<the>","the",det]),v0)
((2,["<bear>","bear",n,sg]),v1)
((2,["<bear>","bear",vblex,pl]),v2)
((2,["<bear>","bear",vblex,fin]),v3)
((3,["<sleeps>","sleep",n,pl]),v4)
((3,["<sleeps>","sleep",vblex,sg,p3]),v5)
\end{lstlisting}

\subsection{Rule ordering}
\label{ssec:ordering}

Ordering and conflicts : the most important differences here

\cite{koskenniemi92}
\begin{quote}Rules in the CG formalism are typically [--] executed as successive groups.
In finite-state syntax, rules are logically unordered.\end{quote}

% VISLCG ad hoc ordering:
% 1. SELECT before REMOVE.
% 2. SELECT rules targetting more preferred tags before 
%   rules targetting less preferred targets.
% 3. REMOVE rules targetting less preferred tags before
 %   rules targetting more preferred targets.
% 4. By order of appearance in rule file.

In CG3, the rules are applied in the order of appearance in the rule file.




% \section*{Acknowledgments}

% Do not number the acknowledgment section. Do not include this section
% when submitting your paper for review.

\bibliographystyle{acl}
\bibliography{cg}


\end{document}
