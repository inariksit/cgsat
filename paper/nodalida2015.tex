%
% File nodalida2015.tex
%
% Contact beata.megyesi@lingfil.uu.se
%
% Based on the instruction file for EACL 2014
% which in turn was based on the instruction files for previous 
% ACL and EACL conferences.

\documentclass[11pt]{article}
\usepackage{nodalida2015}
\usepackage{times}
\usepackage{mathptmx}
%\usepackage{txfonts}
\usepackage{url}
\usepackage{latexsym}
\usepackage{cite}
\usepackage{authordate1-4}
\usepackage{multirow}
\special{papersize=210mm,297mm} % to avoid having to use "-t a4" with dvips 
%\setlength\titlebox{6.5cm}  % You can expand the title box if you really have to

\usepackage{color}
\newcommand{\todo}[1]{{\color{cyan}\textbf{[TODO: }#1\textbf{]}}}
\usepackage{tikz}
\def\checkmark{\tikz\fill[scale=0.4](0,.35) -- (.25,0) -- (1,.7) --
  (.25,.15) -- cycle;}
\newcommand{\Mypm}{\mathbin{\tikz [x=1.4ex,y=1.4ex,line width=.1ex] \draw (0.0,0) -- (1.0,0) (0.5,0.08) -- (0.5,0.92) (0.0,0.5) -- (1.0,0.5);}}%

\title{Constraint Grammar as a SAT problem}

\author{Inari Listenmaa \and Koen Claessen \\
 Chalmers University of Technology, Gothenburg, Sweden \\
 {\tt \{inari,koen\}@chalmers.se} }

\date{\today}

\begin{document}
\maketitle

%\begin{abstract} \end{abstract}


\section{Introduction}

We represent Constraint Grammar (CG) \cite{karlsson1995constraint} 
as a Boolean satisfiability (SAT) problem.
This is attractive from several reasons: formal logic is
well-studied, and serves as an abstract language to reason about the
properties of CG. Constraint rules encoded in logic capture richer
dependencies between the tags than standard CG.

Applying logic to reductionist grammars has been explored earlier by \cite{lager98,lager_nivre01}, but it was never adopted for use.
% ; logic programming was too slow to be used for tagging or parsing. 
Since those works, SAT solving techniques have improved significantly \cite{marques_silva2010}, and they are used in domains such as microprocessor design and computational 
biology---these problems easily match or exceed CG in complexity. 
Due to these advances, we were able to revisit the idea and develop it
further. 

Encoding CG in logic increases computational complexity, 
but as a tradeoff, it could simplify the grammar writing.
Due to representing the rules as implications, the requirement of being unambiguously tagged becomes irrelevant.
One rule corresponds to many logical formulas (each condition being
the target), making the number of the rules potentially smaller.
Ordering can be preserved or discarded; in the latter case, we solve
eventual rule conflicts by finding a solution that discards the least
number of rule applications.
% our experiments show that it is possible to write a small grammar, 
% which, when run with SAT-CG, disambiguates more than the same
%grammar run with VISL CG-3.

We test our implementation by parsing texts in the order of
10,000s--100,000s words, using grammars with hundreds of rules.



%SAT solving is based on a technique called unit propagation:
%a set of complex Boolean clauses is simplified, starting from unit
%clauses, which consist of just a single variable, and working up to a
%solution, where all variables in the set have a True or False value.




% Uncomment for full paper
\section{Related work}
\label{sect:related}

Our work is inspired by \cite{lager98}, which presents constraint rules as a disjunctive logic program,
and \cite{lager_nivre01}, which reconstructs 4 different formalisms from a logical point of view.
\cite{lindberg_eineborg98ilp,asfrent14} use Inductive Logic Programming to learn CG rules from a tagged corpus.
\cite{lager01transformation} discusses ordering of the CG rules.

\section{CG as a SAT problem}
Let us demonstrate our approach with the following example in Spanish.

\begin{verbatim}
"<la>"
        "el" det def f sg
        "lo" prn p3 f sg
"<casa>"
        "casa" n f sg
        "casar" v pri p3 sg
        "casar" v imp p2 sg
\end{verbatim}

The ambiguous passage can be either a noun phrase, \emph{la}\texttt{<det>} \emph{casa}\texttt{<n>} 
`the house'  or a verb phrase \emph{la}\texttt{<prn>}  \emph{casa}\texttt{<v><pri><p3>} `(he/she) marries her'. 
We add the following rules:

\begin{itemize}
\item [] \texttt{REMOVE prn IF (1 n) ;} \\
             \texttt{REMOVE det IF (1 v) ;}
\end{itemize}

Standard CG will apply one\footnote{To be more cautious, we could require the word at position +1 be disambiguated fully (\texttt{1C} instead of \texttt{1}), but in that case, 
neither of the rules would be applied.} of the rules to the word \emph{la}; 
either the one that comes first, or by some other heuristic. 
The second rule will not fire, because it would remove the last reading. 
All readings of \emph{casa} are left untouched by these rules.

The SAT solver performs a search, 
and starts building possible models that satisfy both constraints. 
In addition to the given constraints, we have default rules to emulate
the CG principles: an analysis is true if no rule affects it,
and at least one analysis for each word is true---the notion of
``last'' is not applicable.

With these constraints, we get two solutions. The interaction of the rules regarding \emph{la}  disambiguates the POS of \emph{casa} for free, and ordering of the rules doesn't matter. 

\begin{enumerate}
\item [\texttt{1)}]
\begin{verbatim}
"<la>"
        "el" det def f sg
"<casa>"
        "casa" n f sg
\end{verbatim}
\item [\texttt{2)}]
\begin{verbatim}
"<la>"
        "lo" prn p3 f sg
"<casa>"
        "casar" v pri p3 sg
        "casar" v imp p2 sg
\end{verbatim}
\end{enumerate} 

%Without more context or additional rules we cannot fully disambiguate the passage,
%but unlike with standard CG, we get only legal combinations in one solution.

%If we add a word, \emph{la casa grande} `the big house' 
%and a rule that removes verb if followed by an adjective
%\texttt{REMOVE (v) IF (1 (adj)) ;}
%we will get a unique solution with \texttt{det n adj}.
%Standard CG will also remove the verb reading, 
%but the choice of the word \emph{la} doesn't contribute to

\noindent The most important differences between the traditional and the SAT-based approach are summarised below:

\paragraph{Condition of being unambiguously tagged is irrelevant.}
Rather than waiting for a word to get disambiguated, the SAT solver starts by 
making assumptions (e.g. ``\emph{casa} is a noun'') and working under them,
discarding the assumption if it doesn't lead to a model that satisfies all constraints.

\paragraph{Rules are unordered.}


We have experimented with different strategies:
\begin{itemize}
\item Maximise
% \footnote{Something about local maximum}
%The algorithm commits to a subset of clauses (e.g. \emph{all words must have more than 0 true analyses}), and tries to maximise the number of true variables without changing that subset. So it is a local maximum as regards to the given set of clauses. 
number of rules used---We discarded this strategy, because it was too strong: if a rule leads to a conflict in one case, it was not be applied anywhere.
\item Emulate sequential order---Enter clauses produced by each rule one by one, and assume the solver state reached so far is correct; if a conflict is introduced by new clauses, discard them and move on to next.
\item [\checkmark] Maximise number of rule applications---If there is a conflict, find the smallest number of rule applications to discard so that the conflict is solved.
\end{itemize}
The chosen heuristic produced the best results with examples in the order of tens or hundreds of rules.
Larger rule sets are yet to be tested.

% Finite-state implementations of CG \cite{koskenniemi92} also have logically unordered rules. 
% \todo{Inari: check if Tapanainen 1996 discuss ordering}

\paragraph{Rules disambiguate more.}
Considering our example phrase and rules, standard CG implementation
can only remove readings for the word which is tagged as \texttt{prn} or
\texttt{det}. The SAT-based implementation interprets the rules as
``determiner and verb together are illegal'', and is free to take action that concerns also the word in the condition (\texttt{1 n} or \texttt{1 v}).
This means that the rules are logically flipped: \texttt{REMOVE prn IF
  (1 n)} translates into the same logical formula as  \texttt{REMOVE n
  IF (-1 prn)}. A rule with more conditions can be written as many
rules, each condition taking its turn to be the target of removal or selection. \\

% \begin{itemize}
% \item [] \texttt{REMOVE n IF (-1 prn) ;} \\
%          \texttt{REMOVE v IF (-1 det) ;}
% \end{itemize}

% SAT-based approach gives identical results with both sets of rules,
% whereas the standard CG would remove one reading from \emph{casa} and leave \emph{la} ambiguous.
% Testing this property with more complex rules and larger rule sets remains to be done.

%Rules with more conditions translate into many rules; rules with negation become complements (\texttt{(*)-X} for \texttt{NOT X}). Rules which require the condition to be unambiguously tagged, don't have an equivalent flip in the standard CG.
% We did not expect much practical benefits, save for realising that some rules are bad by having to think further what it implies,
%Out of interest, we manually flipped a small grammar and tested it against the original.
% \todo{The results show  }
%\begin{itemize}
%\item []          \texttt{SL Inf IF (-2 V-MOD) (-1 "not");} \\
%$\rightarrow$\texttt{SL V-MOD   IF (1 "not") (2 INF);} \\
%$\rightarrow$\texttt{SL "not" IF (-1 V-MOD) (1 Inf);}
%\item []           \texttt{SL Foo IF (NOT 1 Bar)} \\
%$\rightarrow$\texttt{SL (*)-Foo if (-1 Bar)}
%\end{itemize}


\noindent These three features influence the way that rules are written. 
We predict that less rules are needed; whether this holds in
the order of thousands of rules remains to be tested.
Getting rid of ordering could ease the task of the grammar writer,
since it removes the need to estimate the best sequence of rule
applications.

However, the results using our approach should not be too different from the
existing ones---our preliminary evaluation against VISL CG-3 shows
promising results in this regard.


\section{Results and evaluation}

\paragraph{Time}

The worst-case complexity of SAT is exponential, whereas the standard
implementations of CG are polynomial, but with advances in SAT solving
techniques, the performance in the average case in practice is much more feasible than in the previous works done in 90s--00s.
We used the open-source SAT solver MiniSat \cite{een04sat}.

Table~\ref{table:time} shows the execution time compared to VISL CG-3. 
The number of rules in the grammar is more significant than the word
count; this holds also for the performance of VISL CG-3.
From the SAT solving side, maximisation is the most costly operation.
Parsing Don Quijote (384,155 words) with the rule set of 261 rules,
the function was called 147,253 times, and with 19 rules, 132,255 times.
%However, with both rule sets, half of the sentences in Don Quijote
%needed less than 6 calls of maximise, and 75 \% of the sentences
%needed less than 14.
The difference in the execution times suggests that there are other
reasons for the worse performance---this is to be expected,
SATCG is currently just a naive proof-of-concept implementation with no optimisations.

\begin{table}
  \centering
  \begin{tabular}{|l|c|c|c|}
     \hline
   \textbf{\# words} & \textbf{\# rules} &  \textbf{SATCG} & \textbf{VISL CG-3} \\ \hline
                384,155  & 19   &  23.1s   & 4.2s\\ %\hline
	        384,155  & 99   &  1m40.6s & 6.1s \\ %\hline
                384,155  & 261  &  2m36.7s & 10.7s \\ \hline
  \end{tabular}
  \caption{Execution times.}
  \label{table:time}
\end{table}

\paragraph{Performance against VISL CG-3}


We took a tagged corpus\footnote{https://svn.code.sf.net/p/apertium/svn/branches/apertium-swpost/apertium-en-es/es-tagger-data/es.tagged} of 21865 words for Spanish, 
and a small constraint grammar\footnote{https://svn.code.sf.net/p/apertium/svn/languages/apertium-spa/apertium-spa.spa.rlx}, produced independently of the authors.
% We ignored substitute rules, but kept subreadings and unification, even though our implementation doesn't handle them.
We only took select and remove rules (no substitute/iff), after which the constraint grammar had 261 rules.
With this setup, we took the text of the tagged corpus with all
ambiguities, and ran both SATCG and VISL CG-3 with the same grammar. 
Treating the corpus as the gold standard, the disambiguation by
SATCG achieves 75.09 \% correct result and VISL CG-3 75.80 \%.
Similar patterns were observed with small (\textless{}20) rule sets
written by the authors; depending on the subset, SATCG and VISL CG-3 had
a difference of at most $\Mypm$ 1.5 \%. 
Introducing rules one by one up to 19, the
performance improved in a very similar rate, with less than 0.5 \%
difference between the systems at each new rule.

% Additional tests could include plugging SATCG into Apertium
% translation pipeline, and comparing the translation quality.

% \paragraph{Removing rules}
% \begin{table}
%   \centering
%   \begin{tabular}{|l|c|c|}
%      \hline
%     \textbf{\# rules} & \textbf{SATCG} & \textbf{VISL CG-3} \\
%     \hline 
% 	19 &  83.15 & 83.33 \\ \hline  % 19 out of 23 preselected by authors
% 	18 &  83.19 & 83.04 \\ \hline  % 18 out of 19
% 	17 &  83.201 & 83.03 \\ \hline   % 17 out of 18/19, no diff
% 	16 &  83.197 & 83.030 \\ \hline  % 16 out of 17/18, no diff
% 	15 &  83.188 & 83.022\\ \hline  % 15 out of 16/17. Tie with 14 for SATCG, chose to retain the rule that gave better results for VISL CG-3.
% 	14 &  83.180 & 83.005 \\ \hline  % Tie with 15 for SATCG.
% 	13 &  83.159 & 82.951 \\ \hline  % 13 out of 15 
% 	12 &  83.134 & 82.884\\ \hline  % 12 out of 14
% 	11 &  83.089 & 82.847\\ \hline  
% 	10 &  83.051 & 82.802\\ \hline  
% 	9  & 82.997 & 82.802 \\ \hline
% 	8  & 82.906 & 82.681 \\ \hline  % REMOVE:r_imp Imp IF (0 Subj) (-1* CnjSub) ; 
% 	7  & 82.743 & 82.594 \\ \hline  % SELECT:s_pro PrnIndep IF (1C VerbFin) ;
% 	6  & 82.556 & 82.398\\ \hline  %  REMOVE:r_v_v Verb IF (-1C Vblex) ;
% 	5  & 82.327 & 82.028 \\ \hline % SELECT:s_adv Adv IF (1 Cnj_Rel_End OR Prep OR Det)
% 	4  & 82.065 & 82.028 \\ \hline % REMOVE:r_vfin_prep VerbFin IF (0 Prep) (1 Inf) 
% 	3  & 81.387 & 81.233 \\ \hline % REMOVE:r_noun N    IF (-1 N OR Prn)
% 	2  & 80.23 & 79.97 \\ \hline % SELECT:s_pri_1 (pri p3) IF (0 (vblex imp p2 sg))
% 	1  & 76.69 & 76.62 \\ \hline 
% 	0  & 70.70 & 70.70 \\ \hline
%   \end{tabular}
%   \caption{Result of removing rules one by one, and how close to the gold standard they get.}
%   \label{table:results_remove}
% \end{table}
% In this experiment, we took 19 rules written by the authors, which give 83.15 \% correct results with SATCG and 83.33 \% with VISL CG-3.
% Then we removed rules one by one; after getting a score for $n$ rules, we took the combination of $n-1$ rules for which SATCG gives the best score, and we ran the rule set with VISL CG-3 for comparison. Table~\ref{table:results_remove} shows the percentages compared to the gold standard for both systems.

\section{Conclusions}

SAT-solvers are nowadays powerful enough to be used for dealing with Constraint Grammar. A logic-based approach to CG has possible advantages over more traditional approaches; a SAT-solver may disambiguate more words, and may do so more precisely. Also, the SAT-solver requires less rules, and these rules are simpler. The ordering of rules is something we found was incompatible with a logic-based approach. We compensate this by maximising rule applications. Whether this is an advantage or disadvantage remains to be seen. Our initial experimental results are promising.


% \section*{Acknowledgments}

% Do not number the acknowledgment section. Do not include this section
% when submitting your paper for review.

\bibliographystyle{acl}
\bibliography{cg}


\end{document}
