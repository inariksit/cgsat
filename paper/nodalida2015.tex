%
% File nodalida2015.tex
%
% Contact beata.megyesi@lingfil.uu.se
%
% Based on the instruction file for EACL 2014
% which in turn was based on the instruction files for previous 
% ACL and EACL conferences.

\documentclass[11pt]{article}
\usepackage{nodalida2015}
\usepackage{times}
\usepackage{mathptmx}
%\usepackage{txfonts}
\usepackage{url}
\usepackage{latexsym}
\usepackage{cite}
\usepackage{authordate1-4}
\usepackage{multirow}
\special{papersize=210mm,297mm} % to avoid having to use "-t a4" with dvips 
%\setlength\titlebox{6.5cm}  % You can expand the title box if you really have to

\usepackage{color}
\newcommand{\todo}[1]{{\color{cyan}\textbf{[TODO: }#1\textbf{]}}}
\usepackage{tikz}
\def\checkmark{\tikz\fill[scale=0.4](0,.35) -- (.25,0) -- (1,.7) -- (.25,.15) -- cycle;}
\title{Exploring Constraint Grammar as a SAT problem}

\author{Inari Listenmaa \and Koen Claessen \\
  Chalmers University of Technology, Gothenburg, Sweden \\
  {\tt \{inari,koen\}@chalmers.se} }

\date{\today}

\begin{document}
\maketitle

%\begin{abstract}
%We implement Constraint Grammar as a Boolean satisfiability problem.
%We experiment with different strategies to maximise the number of instances to apply rules;
%this will also lead to different orderings of the rules and ways of solving conflicts.
%\end{abstract}


\section{Introduction}
%Constraint Grammar (CG) was first introduced by \cite{karlsson1995constraint}. 
%It disambiguates output by morphological analyser, 
%and has been used for many tasks in computational linguistics, such as POS tagging,
%surface syntax and machine translation.
%The current state-of-the-art parser VISL CG-3 is implemented (...);
%in addition there are a number of finite-state based implementations \cite{nemeskey14}.

We represent Constraint Grammar \cite{karlsson1995constraint} 
as a Boolean satisfiability (SAT) problem.
%This increases the computational complexity, 
%but we are exploring ways in which it could simplify the grammar writing:
%our experiments show that less rules are needed, and features like ordering of the rules
%and requirement of being unambiguously tagged become irrelevant.
Let us demonstrate our approach with the following example in Spanish.

\begin{verbatim}
"<la>"
        "el" det def f sg
        "lo" prn p3 f sg
"<casa>"
        "casa" n f sg
        "casar" v pri p3 sg
        "casar" v imp p2 sg
\end{verbatim}

The ambiguous passage can be either a noun phrase, \emph{la\texttt{:det} casa\texttt{:n}} 
`the house'  or a verb phrase \emph{la\texttt{:prn} casa\texttt{:v pri p3}} `(he/she) marries her'. 
We add the following rules:

\begin{itemize}
\item [] \texttt{REMOVE prn IF (1 n) ;} \\
             \texttt{REMOVE det IF (1 v) ;}
\end{itemize}

Standard CG will apply one\footnote{To be more cautious, we could require the word at position +1 be disambiguated fully (\texttt{1C} instead of \texttt{1}), but in that case, 
neither of the rules would be applied.} of the rules to the word \emph{la}; 
either the one that comes first, or by some other heuristic. 
The second rule will not fire, because it would remove the last reading. 
All readings of \emph{casa} are left untouched by these rules.

SAT solver 
performs a search, 
and starts building possible models that satisfy both constraints. 
In addition to the given constraints, we have default rules to emulate
the CG principles: an analysis is true if no rule affects it,
and at least one analysis for each word is true---the notion of ``last'' is not applicable.
With these constraints, we get two solutions: \texttt{det, n} and \texttt{prn, v pri p3, v imp p2}. 
The interaction of the rules regarding \emph{la}  disambiguates the POS of \emph{casa} 
for free, and ordering of the rules doesn't matter. 
%Without more context or additional rules we cannot fully disambiguate the passage,
%but unlike with standard CG, we get only legal combinations in one solution.

%If we add a word, \emph{la casa grande} `the big house' 
%and a rule that removes verb if followed by an adjective
%\texttt{REMOVE (v) IF (1 (adj)) ;}
%we will get a unique solution with \texttt{det n adj}.
%Standard CG will also remove the verb reading, 
%but the choice of the word \emph{la} doesn't contribute to

The most important differences between the traditional and the SAT-based approach are summarised below:

\paragraph{Condition of being unambiguously tagged is irrelevant.}
Rather than waiting for a word to get disambiguated, the SAT solver starts by 
making assumptions (e.g. ``\emph{casa} is a noun'') and working under them,
discarding the assumption if it doesn't lead to a model that satisfies all constraints.

\paragraph{Rules are unordered.}
A typical CG rule sequence starts with rules that are either very common, or have a complex condition to fulfil.
Stronger rules appear towards the end, to be applied if none of the weaker rules has. 
%For a rule set such as following, this order is crucial:

%\begin{itemize}
%\item [] \texttt{SELECT Year IF (-1 Seasons OR Months) ;} \\
 %            \texttt{REMOVE Year ;}
%\end{itemize}

To replace the sequential order, we have experimented with different strategies:
\begin{itemize}
\item Maximise number of rules used---We discarded this strategy, because it was too strong: if a rule leads to a conflict in one case, it was not be applied anywhere.
\item Emulate sequential order---Enter clauses produced by each rule one by one, and assume the solver state reached so far is correct; if a conflict is introduced by new clauses, discard them and move on to next.
\item [\checkmark] Maximise number of rule applications---If there is a conflict, find the smallest number of rule applications to discard so that the conflict is solved.
\end{itemize}

%combined with the following
%\begin{itemize}
%\item Maximise number of true analyses---Without this heuristic, both solutions to our example %\emph{la casa} are as likely. With this heuristic,  the example would be disambiguated as \texttt{prn v}, because there happens to be two analyses with \texttt{v}, and choosing that, 3 analyses are true, since there is no rule to disambiguate between imperative and indicative.
% We want analyses to be true by default, but still disambiguate as much as possible.
%We have a rule to maximise the number of true analyses; then we take a solution that has the least. 
%\item Minimise number of true analyses---To combat the previous, we want each solution to contain as many true analyses as possible, to ensure that analyses untouched by rules are kept true. %After having multiple solutions which all ensure , we pick the one that disambiguates the most.
%\end{itemize}


\paragraph{Rules disambiguate more.}
% Rules interact with each other and disambiguate more than same rules in the standard implementation. 
In standard CG, the rules only tell to remove readings for the word which is tagged as 
\texttt{prn} or \texttt{det}. The SAT-based implementation interprets the rules as ``noun and pronoun together are illegal'', and is free to take action that concerns also the following word.
This means that the rules can be flipped: compare the rules below to the ones in the example.

\begin{itemize}
\item [] \texttt{REMOVE n IF (-1 prn) ;} \\
             \texttt{REMOVE v IF (-1 det) ;}
\end{itemize}

SAT-based approach gives identical results with both sets of rules,
whereas the standard CG would remove one reading from \emph{casa} and leave \emph{la} ambiguous.
Testing this property with more complex rules and larger rule sets than remains to be done.

%Rules with more conditions translate into many rules; rules with negation become complements (\texttt{(*)-X} for \texttt{NOT X}). Rules which require the condition to be unambiguously tagged, don't have an equivalent flip in the standard CG.
% We did not expect much practical benefits, save for realising that some rules are bad by having to think further what it implies,
%Out of interest, we manually flipped a small grammar and tested it against the original.
% \todo{The results show  }
%\begin{itemize}
%\item []          \texttt{SL Inf IF (-2 V-MOD) (-1 "not");} \\
%$\rightarrow$\texttt{SL V-MOD   IF (1 "not") (2 INF);} \\
%$\rightarrow$\texttt{SL "not" IF (-1 V-MOD) (1 Inf);}
%\item []           \texttt{SL Foo IF (NOT 1 Bar)} \\
%$\rightarrow$\texttt{SL (*)-Foo if (-1 Bar)}
%\end{itemize}

The heuristics worked with examples in the order of tens of rules, but moving to hundreds of rules, the rules interact more, the results are more mixed. 
Rule sets in order of thousands of rules remains to be tested.

% Finite-state implementations of CG \cite{koskenniemi92} also have logically unordered rules. 
% \todo{Inari: check if Tapanainen 1996 discuss ordering}

These three features are likely to change the way that rules are written. \todo{more speculation, examples}


\section{Related work}
\label{sect:related}

Our work is inspired by \cite{lager98}, which presents constraint rules as a disjunctive logic program,
and \cite{lager_nivre01}, which reconstructs 4 different formalisms from a logical point of view.
\cite{lindberg_eineborg98ilp,asfrent14} use Inductive Logic Programming to learn CG rules from a tagged corpus.
\cite{lager01transformation} discusses ordering of the CG rules.


\section{Results and evaluation}

\paragraph{Time}

SAT is NP-complete, whereas the standard implementations of CG are polynomial, 
but with advances in SAT solving techniques, 
the performance is much more feasible than in the previous works done in 90s--00s.
We used the open-source SAT solver MiniSat \cite{een04sat}.
%We do foreach sentence: foreach rule; 
%whereas VISL CG3 has foreach rule: foreach sentence.
%the size of the SAT problem is dependent of the sentence (???)

Table~\ref{table:time} shows the execution time compared to VISL CG3.
SATCG is just a first proof-of-concept implementation and can be optimised.

\begin{table}
  \centering
  \begin{tabular}{|l|l|c|c|}
     \hline
	        \textbf{Text} & \textbf{\# words, rules} &  \textbf{SATCG} & \textbf{VISLCG3} \\ \hline
    \multirow{2}{*}{\textbf{Don Quijote}} & 186460 words  &  2m53.7s & 10.7s \\ 
		                                           & 261 rules & & \\ \hline
    \multirow{2}{*}{\textbf{Pride and}}  & 153386 words& 55.7s  & 4.8s \\ 
                         {\textbf{Prejudice}}                   & 99 rules & \\ \hline
  \end{tabular}
  \caption{Comparing execution times between SATCG and VISLCG3.}
  \label{table:time}
\end{table}

\paragraph{Performance against VISL CG3}


We took a tagged gold standard corpus\footnote{https://svn.code.sf.net/p/apertium/svn/branches/apertium-swpost/apertium-en-es/es-tagger-data/es.tagged} of 21865 words for Spanish, 
and a small constraint grammar\footnote{https://svn.code.sf.net/p/apertium/svn/languages/apertium-spa/apertium-spa.spa.rlx}, produced independently of the authors.
% We ignored substitute rules, but kept subreadings and unification, even though our implementation doesn't handle them.
We only took select and remove rules (no substitute/iff), after which the constraint grammar had 261 rules.
With this setup, we took the text of the tagged corpus with all ambiguities, and ran both VISLCG3 and our implementation with the same grammar. 

% small constraint grammars (\textless{}500 rules) for Spanish and English, and disambiguated a number of texts from Project Gutenberg. Table~\ref{table:results} shows how much the results agree with the result given by VISL CG3 parser with the same grammar and text.

%\begin{table}
% \centering
%  \begin{tabular}{|l|c|c|}
%     \hline
%    \textbf{Text} & \textbf{Words} & \textbf{\% same analyses} \\
%    \hline
%	Pride and Prejudice & 153386 & 90.6 \\  
%    \hline
%       Don Quijote & 186460 & 96.3  \\ 
%     \hline
%  \end{tabular}
%  \caption{Comparing analyses between SATCG and VISLCG3.}
%  \label{table:results}
%\end{table}
% The tendency of needing less rules to disambiguate shows also in a larger context with more complex rules. 
% SATCG disambiguated more 17 \% of the cases for Pride and Prejudice, and 28 \% of the cases for Don Quijote. \todo{Analysis of those cases}

\paragraph{Removing rules}
\begin{table}
  \centering
  \begin{tabular}{|l|c|c|}
     \hline
    \textbf{\# rules} & \textbf{SATCG} & \textbf{VISLCG3} \\
    \hline 
	19 &  83.15 & 83.33 \\ \hline  % 19 out of 23 preselected by authors
	18 &  83.19 & 83.04 \\ \hline  % 18 out of 19
	17 &  83.201 & 83.03 \\ \hline   % 17 out of 18/19, no diff
	16 &  83.197 & 83.030 \\ \hline  % 16 out of 17/18, no diff
	15 &  83.188 & 83.022\\ \hline  % 15 out of 16/17. Tie with 14 for SATCG, chose to retain the rule that gave better results for VISLCG3.
	14 &  83.180 & 83.005 \\ \hline  % Tie with 15 for SATCG.
	13 &  83.159 & 82.951 \\ \hline  % 13 out of 15 
	12 &  83.134 & 82.884\\ \hline  % 12 out of 14
	11 &  83.089 & 82.847\\ \hline  
	10 &  83.051 & 82.802\\ \hline  
	9  & 82.997 & 82.802 \\ \hline
	8  & 82.906 & 82.681 \\ \hline  % REMOVE:r_imp Imp IF (0 Subj) (-1* CnjSub) ; 
	7  & 82.743 & 82.594 \\ \hline  % SELECT:s_pro PrnIndep IF (1C VerbFin) ;
	6  & 82.556 & 82.398\\ \hline  %  REMOVE:r_v_v Verb IF (-1C Vblex) ;
	5  & 82.327 & 82.028 \\ \hline % SELECT:s_adv Adv IF (1 Cnj_Rel_End OR Prep OR Det)
	4  & 82.065 & 82.028 \\ \hline % REMOVE:r_vfin_prep VerbFin IF (0 Prep) (1 Inf) 
	3  & 81.387 & 81.233 \\ \hline % REMOVE:r_noun N    IF (-1 N OR Prn)
	2  & 80.23 & 79.97 \\ \hline % SELECT:s_pri_1 (pri p3) IF (0 (vblex imp p2 sg))
	1  & 76.69 & 76.62 \\ \hline 
	0  & 70.70 & 70.70 \\ \hline
  \end{tabular}
  \caption{Result of removing rules one by one, and how close to the gold standard they get.}
  \label{table:results_remove}
\end{table}
In this experiment, we took 19 rules written by the authors, which give 83.15 \% correct results with SATCG and 83.33 \% with VISLCG3.
Then we removed rules one by one; after getting a score for $n$ rules, we took the combination of $n-1$ rules for which SATCG gives the best score, and we ran the rule set with VISLCG3 for comparison. Table~\ref{table:results_remove} shows the percentages compared to the gold standard for both systems.



% \section*{Acknowledgments}

% Do not number the acknowledgment section. Do not include this section
% when submitting your paper for review.
% Francis Tyers

\bibliographystyle{acl}
\bibliography{cg}


\end{document}
