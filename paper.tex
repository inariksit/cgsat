\documentclass[a4paper, 11pt]{article}
\topmargin-2.0cm

\usepackage{fancyhdr}
\usepackage{lastpage}
\usepackage{extramarks}
\usepackage{pagecounting}
\usepackage{color}
\usepackage[round]{natbib}
\usepackage{lipsum}
\usepackage{enumerate}
%\usepackage{tipa}
%\usepackage{gb4e}
\usepackage{graphicx}
\usepackage{amsmath, amsthm}
%\usepackage{colortbl}
\usepackage{caption}
\usepackage[a4paper]{geometry}
\usepackage{courier}
\usepackage{listings}
\lstset{
         basicstyle=\footnotesize\ttfamily, 
         numberstyle=\tiny,          
         numbersep=5pt,             
         tabsize=2,                
         extendedchars=true,      
         breaklines=true,        
         showspaces=false,      
         showtabs=false,       
         xleftmargin=17pt,
         framexleftmargin=17pt,
         framexrightmargin=5pt,
         framexbottommargin=4pt,
         showstringspaces=false 
 }
 \lstloadlanguages{
         Haskell
 }

\usepackage{url}

\advance\oddsidemargin-0.35in
\advance\evensidemargin-0.65in
\textheight9.5in
\textwidth6.5in

\newcommand{\hsp}{\hspace*{\parindent}}
\definecolor{gray}{rgb}{0.4,0.4,0.4}

\newcommand\blfootnote[1]{
  \begingroup
  \renewcommand\thefootnote{}\footnote{#1}
  \addtocounter{footnote}{-1}
  \endgroup
}

%\sectionfont{\large}

\begin{document}
\lstset{language=Haskell}
\pagestyle{fancy}
\lhead{\textcolor{gray}{Inari Listenmaa}}
\chead{\textcolor{gray}{\bf Grammar Formalisms}}
\rhead{\textcolor{gray}{LP2 2014}}
\lfoot{\textcolor{gray}{}}
\rfoot{\thepage}
\renewcommand{\headrulewidth}{0.5pt} 
\renewcommand{\footrulewidth}{0.5pt} 
\fancyfoot[C]{\footnotesize \textcolor{gray}{}} 

\centerline{ {\Large \bf Constraint Grammar} }
\vspace*{0.2cm}

\section{Introduction}

In this course paper, I describe the implementation of Constraint Grammar in Haskell, using a SAT solver to find a solution that fits all the constraints.

First I describe the existing implementations, followed by my own SAT-based solution.

\section{Constraint Grammar}

Constraint grammar was first introduced by \cite{KarlssonTODO}. 
It a tool for disambiguating output by morphological analyser.

\section{Standard implementation(s)}

The first implementation, CG-1, was introduced in \cite{KarlssonTODO}. Karlsson describes the implementation as ``ad hoc`` and ``without any deeper technical contributions to parsing theory``, and adds that ideally, CG should be implemented with finite state methods.
CG-2, implemented by \cite{tapanainen1996}, used finite state technology. 
The latest version, CG-3, was implemented by \cite{TODO} without finite state technology. The implementation introduced new properties, such as ordering of the rules.

\begin{quote}
[--] the Lisp, C, and C++ implementations of the Constraint Grammar Parser are all fairly ad hoc rule interpreters without any deeper technical contributions to parsing theory. These implementations are not based on the use of well understood and theoretically sound parsing algorithms. Rather, they could be characterized as situation-action parsing programs. 
\end{quote}

\subsection{Workflow}

All implementations (TODO check) go through the input many times and apply all possible rules each time. 
After one pass, it may be possible to apply some rules that didn't fire the last time.


\subsection{Resolving conflicts}

For CG-1 and CG-2, ... sections? TODO

In CG-3, the rules are ordered. In case of a conflict, the rule that is described first is given priority.


\section{SAT based implementation}

Unit propagation magic!

\subsection{Data types}

% \begin{verbatim}
 \begin{lstlisting}
data Rule = Remove [Tag] Condition | Select [Tag] Condition

data Condition = C Position [Tag]
               | NOT Condition
               | AND Condition Condition
               | OR Condition Condition 
\end{lstlisting}

The data type for Rule is either remove or select a list of tags, with condition(s).
For Condition, there are basic conditions (position and a list of context tags) and combined conditions, with constructors `NOT', `AND' and `OR'. For instance, the following rule is read as ``remove tag for any verb, if the lemma is \emph{bear} or there is a determiner or an article anywhere to the left''.

\begin{lstlisting}
verb = [V, V2, V3, VV, V2V, VS]
Remove verb (OR (C (Exactly 0)  [Lem "bear"])
                (C (AtLeast -1) [Det,Art])
            )
\end{lstlisting}

There is no special constructor for empty condition (ie. remove/select tag everywhere), but an empty tag list in a condition, i.e. \texttt{C \_ []} is assumed to mean that.

The data type for position can be exact n places or at least n places. The number 0 means the word itself, negative number means \emph{n} positions to the left and positive \emph{n} positions to the right.

\begin{lstlisting}
data Position = Exactly Integer | AtLeast Integer
 \end{lstlisting}


%If you want references on a separate page, uncomment the following command.
%\clearpage

%\begin{small}
%\bibliographystyle{plainnat}
%\bibliography{references}
%\end{small}

\end{document}

